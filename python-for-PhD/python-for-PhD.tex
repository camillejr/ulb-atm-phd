\documentclass[10pt,twocolumn]{article}
\usepackage{geometry}
\geometry{verbose,headsep=3cm,tmargin=2.5cm,bmargin=2.5cm,lmargin=2.0cm,rmargin=2.0cm}
\usepackage{graphicx}
\usepackage{xcolor}
\usepackage[font=small]{caption}
\usepackage{amsmath,amssymb,latexsym}
\usepackage{marvosym}
\usepackage{url}
\usepackage{lipsum}
\usepackage{bm}
\usepackage{float}
\usepackage[english]{babel}
\usepackage{hyperref}
\usepackage{epsf}
\usepackage{float}
\usepackage{mathpazo}
\usepackage{pifont}
\usepackage{wrapfig}
\usepackage{multicol}
\usepackage{enumitem}
\usepackage{xcolor}
\usepackage{framed}
\usepackage[utf8]{inputenc}
\graphicspath{{DWGs/}}
\usepackage{framed}
\usepackage{textcomp}

% Document font:
\usepackage{charter}

\newcommand{\highlight}[1]{%
  \colorbox{orange!50}{$\displaystyle#1$}}
% Default fixed font does not support bold face
\DeclareFixedFont{\ttb}{T1}{txtt}{bx}{n}{10} % for bold
\DeclareFixedFont{\ttm}{T1}{txtt}{m}{n}{10}  % for normal

% Custom colors
\usepackage{color}
\definecolor{deepblue}{rgb}{0,0,0.5}
\definecolor{deepred}{rgb}{0.6,0,0}
\definecolor{deepgreen}{rgb}{0,0.5,0}

\usepackage{listings}

% Python style for highlighting
\newcommand\pythonstyle{\lstset{
language=Python,
basicstyle=\ttm,
otherkeywords={self},             % Add keywords here
keywordstyle=\ttb\color{deepblue},
emph={MyClass,__init__},          % Custom highlighting
emphstyle=\ttb\color{deepred},    % Custom highlighting style
stringstyle=\color{deepgreen},
frame=tb,                         % Any extra options here
showstringspaces=false            % 
}}


% Python environment
\lstnewenvironment{python}[1][]
{
\pythonstyle
\lstset{#1}
}
{}

% Python for external files
\newcommand\pythonexternal[2][]{{
\pythonstyle
\lstinputlisting[#1]{#2}}}

% Python for inline
\newcommand\pythoninline[1]{{\pythonstyle\lstinline!#1!}}

\begin{document}

%%% HEADER --------------------------------------------------------------
% ------------------------------------------------------------------------

\twocolumn[{
\begin{@twocolumnfalse}

  \begin{center}
%\textcolor{lgray}
    \vskip-5em

    \hfill
    \fontsize{10}{10}\selectfont {\textit{Bruxelles, December 2018}}
    \vskip2ex
	\vspace{5ex}
    \fontsize{24}{10}\selectfont {Python for PhD}
      \vspace{1ex}
  \noindent%
    
\vskip1ex

{\rule{\textwidth}{0.5pt}}

  \end{center}
  
\vspace{8mm}

\end{@twocolumnfalse}
}]

%%% HEADER END -----------------------------------------------------------
% ------------------------------------------------------------------------

\vspace{10mm}

\setlength{\parindent}{0cm}

\fontsize{14}{10}\selectfont {Kamila Zdybał}

\vspace{2mm}

\fontsize{8}{10}\selectfont {\textit{Université libre de Bruxelles, kamila.zdybal@ulb.ac.be}}

\fontsize{8}{10}\selectfont {\textit{camillejr.github.io/science-docs, kamila.zdybal@gmail.com}}

\section*{Preface}

This is a collection of Python tips to boost your everyday PhD life.

\,\,

This document is still in preparation. Please feel free to contact me with any suggestions, corrections or comments.

\tableofcontents

\section{Numpy cheatsheet}


\begin{python}
import numpy as np
\end{python}


\pythoninline{np.arange(start, stop, step)}

\pythoninline{np.linspace(start, stop, num_points)} - distribution 

Numpy arrays of random numbers:

\pythoninline{np.random.rand(rows, cols)} – distribution from $[0,1]$ of size $(\text{row} \times \text{col})$

\section{Inputing external code}

\pythonexternal{test.py}

\section{Scientific computing cheatsheet}

\subsection{Noise generation}


numpy.random.normal


\end{document}
