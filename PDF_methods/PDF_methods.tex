\documentclass[10pt,twocolumn]{article}
\usepackage{geometry}
\geometry{verbose,headsep=3cm,tmargin=2.5cm,bmargin=2.5cm,lmargin=2.0cm,rmargin=2.0cm}
\usepackage{graphicx}
\usepackage{xcolor}
\usepackage[font=small]{caption}
\usepackage{cleveref}
\usepackage{amsmath,amssymb,latexsym}
\usepackage{marvosym}
\usepackage{url}
\usepackage{lipsum}
\usepackage{bm}

\begin{document}

\twocolumn[{
\begin{@twocolumnfalse}


  \begin{center}

    \vskip-3em

    \hfill
    \textbf{Bruxelles, September 2018}

    \rule{\textwidth}{0.5pt}
    \vskip2ex

    {\Large\textbf{PDF methods for turbulent reactive flows}}
  
    \vspace{2ex}

      \vspace{1ex}
   
	\texttt{https://camillejr.github.io/science-docs/}
          
  \noindent%
    
\vskip1ex

\rule{\textwidth}{0.5pt}

  \end{center}
  
\vspace{8mm}

\end{@twocolumnfalse}
}]

\tableofcontents

\vspace{10mm}

\setlength{\parindent}{0cm}

\section{Introduction}

This are notes on PDF methods.

\section{PDF methods theory}

The PDF (Probability Density Functions) methods























\appendix

\section{APP1} \label{app:A}

\section{APP2} \label{app:B}

\thebibliography{}

\bibitem{Pope} S.B. Pope \textit{PDF methods for turbulent reactive flows}
Proo. Energy Combust. Sci. 1985, Vol. 11, pp. i 19 192 \label{bib:pope}


\end{document}
