\documentclass[10pt,twocolumn]{article}
\usepackage{geometry}
\geometry{verbose,headsep=3cm,tmargin=2.5cm,bmargin=2.5cm,lmargin=2.0cm,rmargin=2.0cm}
\usepackage{graphicx}
\usepackage{xcolor}
\usepackage[font=small]{caption}
\usepackage{amsmath,amssymb,latexsym}
\usepackage{marvosym}
\usepackage{url}
\usepackage{lipsum}
\usepackage{bm}
\usepackage{float}
\usepackage[english]{babel}
\usepackage{hyperref}
\usepackage{epsf}
\usepackage{float}
\usepackage{mathpazo}
\usepackage{pifont}
\usepackage{wrapfig}
\usepackage{multicol}
\usepackage{enumitem}
\usepackage{xcolor}
\usepackage{framed}
\usepackage[utf8]{inputenc}
% Document font:
\usepackage{charter}

\begin{document}

\twocolumn[{
\begin{@twocolumnfalse}

  \begin{center}
%\textcolor{lgray}
    \vskip-5em

    \hfill
    \fontsize{10}{10}\selectfont {\textit{Bruxelles, March 2019}}
    \vskip2ex
	\vspace{5ex}
    \fontsize{20}{10}\selectfont {Notes on Gaussian Process Regression}
      \vspace{1ex}
      
      \fontsize{16}{10}\selectfont {(with Python examples)}
  \noindent%
    
\vskip1ex

{\rule{\textwidth}{0.5pt}}

  \end{center}
  
    \fontsize{7}{10}\selectfont {This work is licensed under the Creative Commons Attribution-NonCommercial-ShareAlike 4.0 International (CC BY-NC-SA 4.0) license.}

\vspace{6mm}

\end{@twocolumnfalse}
}]

%%% HEADER END -----------------------------------------------------------
% ------------------------------------------------------------------------

\vspace{10mm}

\setlength{\parindent}{0cm}

\fontsize{14}{10}\selectfont {Kamila Zdybał}

\vspace{2mm}

\fontsize{8}{10}\selectfont {\textit{Université libre de Bruxelles, kamila.zdybal@ulb.ac.be}}

\fontsize{8}{10}\selectfont {\textit{camillejr.github.io/science-docs, kamila.zdybal@gmail.com}}

\section*{Preface}

\,\,

This document is still in preparation. Please feel free to contact me with any suggestions, corrections or comments.

\section*{Keywords}

\textit{Gausssian Process Regression (GPR), regression, covariance}

\tableofcontents

\section{Introduction}




\section{Distribution over functions} \label{sec:dist_over_fun}



\section{Probability}

Two words that will be often used in GP are prior and posterior. The prior and posterior probabilities in GP are often linked to functions and it is important to understand their notion. 

The \textbf{\textit{prior probability}} is the probability we assign to a specific object (event) before we collect further evidence or make observations that can change it. It can be viewed as an "initial guess" or "initial belief" about the given system. The \textbf{\textit{posterior probability}} is the updated prior probability that takes into account the new observations or evidence. Conceptually, we may write the following:

\begin{equation}
\text{prior distribution} + \text{data} = \text{posterior distribution}
\end{equation}

As explained in section \ref{sec:dist_over_fun}, in GP we encounter probabilities associated to functions. The prior will be the functions that we believe will fit our data well before we actually attempt fitting these functions to new observation points. After the observations have been made, the functions get the "update" to better fit the newly arrived points and these functions that are left will become our posterior.


\section{Covariance matrix}

Let's think about a dot product between two vectors:

\begin{equation}
\text{dot}(x_i, x_j) = |x_i| |x_j| \text{cos}(\phi)
\end{equation}

It describes the amount of projection of vector $x_i$ onto $x_j$ (or vice versa) and can be useful when we need to know how much one vector points in the direction of the other. 

Now imagine that you have a data set $\mathbf{X}$ with $n$ vectors (features), and you would like to know what is the dot product of every possible pair drawn from these vectors. In other words, you would like to know how correlated are all vectors with each other. You can achieve this "global" dot product by multiplying:

\begin{equation}
\mathbf{S} = \mathbf{X}^T \mathbf{X} 
\end{equation}

the result $\mathbf{S}$ is called the \textit{covariance matrix}. Notice that every entry $(i,j)$ in this matrix is a dot product $\text{dot}(x_i, x_j)$ and it is:

\begin{equation}
\text{dot}(x_i, x_j) = \text{cov}(x_i, x_j) \,\,\, \text{for} \,\,\, i \neq j
\end{equation}

\begin{equation}
\text{dot}(x_i, x_j) = \text{var}(x_i, x_j) \,\,\, \text{for} \,\,\, i = j
\end{equation}

The covariance matrix is symmetric due to symmetry: $\text{dot}(x_i, x_j) = \text{dot}(x_j, x_i)$.

\section{Covariance kernels}

The \textit{covariance kernel} is essentialy a function that populates the covariance matrix. This makes our life easier, since first, this matrix might be huge and second, we can easily implement the underlying structure to the covariance.

The covariance kernel has to be designed such that there is symmetry: $K(x_i, x_j) = K(x_j, x_i)$.\footnote{In \cite{Scaife}, it has been said that the covariance can vary in different directions which made me wonder...}

\subsection{Examples}

Squared exponential kernel:

\begin{equation}
K(x_i, x_j) = h^2 \exp(\frac{- (x_i - x_j)^2}{\lambda^2})
\end{equation}


\section{The meaning of hyperparameters}



\section{Building your GPR in Python}











\thebibliography{}

\bibitem{Rasmussen} C. E. Rasmussen, C. Williams, \textit{Gaussian Process for Machine Learning}, 2006

\bibitem{GP_for_TM} S. Roberts, M. Osborne, M. Ebden, S. Reece, N. Gibson, S. Aigrain \textit{Gaussian Processes for Timeseries Modelling}, 2012

\bibitem{Scaife} A. Scaife, \textit{Machine Learning: Gaussian Process Modelling in Python}, an online lecture



 \label{bib:pope}


\end{document}
